\documentclass[11pt]{article}
\usepackage[T1]{fontenc}
\usepackage{titling}
\usepackage[bottom=10mm]{geometry}
\usepackage{graphicx} % Required for inserting images
\usepackage[htt]{hyphenat}

\setlength{\droptitle}{-5em}

\title{Implementation of closed-loop control for UUV}
\author{Jin Rhee}
\date{18 October 2024}

\begin{document}

\maketitle
\thispagestyle{empty}
% Background and motivation
\paragraph{Background} This report describes work for the B1 Scientific Computing practical. The aim is to develop a closed-loop controller for an UUV given mission data and submarine dynamics.
Specific tasks include: 1. to import mission data from an external \texttt{.csv} file, and
2. to implement and test a PD feedback controller.

% List of changes made to codebase
\paragraph{Key changes}
Changes made to files within \texttt{/uuv\_mission} include:
\par 1. implement \texttt{Mission.from\_csv()} to import \texttt{.csv} mission data,
\par 2. implement a PD (proportional-derivative) controller as the \texttt{Controller} class,
\par 3. complete \texttt{ClosedLoop.simulate()}, and
\par 4. add \texttt{Trajectory.get\_traj\_error()} which returns trajectory error for gain tuning.
\par
Changes to other files include a new bash script that conveniently imports Python modules, and a
test unit within \texttt{notebooks/demo.ipynb} which tests and determines optimal control gains with the
lowest trajectory error.

%1. adding installation script for convenient import of python modules
%2. implementing PD controller as Controller class in separate module (uuv_mission/control.py)
%3. completing simulate() function within ClosedLoop class (uuv_mission/dynamic.py)
%. adding get_traj_error() function within Trajectory class to quantify trajectory error for testing control gain

% Explain design choices
\paragraph{Design choices}
1. The controller is implemented as a Class rather than a function such that future iterations that use different 
control schemes can be implemented as a class method. It has three internal variables: 
\texttt{prev\_error} (initially set to 0 at time zero) and two gains (set to values in handout). Functions \texttt{set\_gains()} and \texttt{set\_prev\_error()} allow the user to 'reset' 
a class instance for reuse instead of constructing a new instance.
\par 2. The function \texttt{get\_traj\_error()} returns the root of squared error sums for a trajectory as to 'normalize'
possibly negative error values. This is sampled and averaged over a number of iterations for one control gain pair
to compare with another.
\par 3. The \texttt{./install\_requirements} script was created for the user to easily install Python dependencies.
While arguably simple in its current state, other processes (such as virtual env setup) could also be automated by this
 script in future iterations.

% Difficulties, future improvements
\paragraph{Difficulties and future work}
One inconvenience when implementing the trajectory error function involves the error state, which is internally
computed within a \texttt{ClosedLoop} instance at each timestep, which is inaccessible. Thus the error (reference - depth)
value must be recomputed within \texttt{Trajectory.get\_traj\_error()} at each timestep, representing an inefficiency.
Future work could solve this by restructuring the code.
\par
Other work could involve improving the setup script,
implementing other modes of control (PID, MPC, etc.) within the \texttt{Controller} class,
implementing other heuristic tuning methods (i.e. Ziegler-Nichols method),
and implementing such tuning methods as a separate function or class.
\end{document}